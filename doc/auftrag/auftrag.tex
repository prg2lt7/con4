\documentclass[a4paper, 10pt, fleqn]{article}

\usepackage{layout}

\author{Pascal Häfliger, Fabian Niederberger, Marc Nussbaumer, Daniel Winz}
\title{Projektauftrag "Vier Gewinnt"}

\begin{document}
\maketitle
Ich brauche ein Java-Programm, mit dem ich Vier Gewinnt spielen kann. 

\section*{Anforderungen}
Ich will: 
\begin{itemize}
    \item Vier Gewinnt gegen den Computer spielen
    \item Im Netz nach menschlichen Gegenspielern suchen und einen 
        Gegenspieler auswählen. Der Gegenspieler darf den ersten 
        Spielstein setzen. 
    \item Wenn ich gegen den Computer spiele, mchte ich das Spiel 
        unterbrechen, speichern und später mit dem gleichen Spielstand 
        weiterspielen können. 
\end{itemize}
Beim Spielen sollten die Regeln von Vier Gewinnt gelten. 

\section*{Zusätzliche Anforderungen}
Es wäre schön, wenn ich für den Gegenspieler keine IP-Adresse angeben 
müsste, sondern mögliche Gegenspieler im Netzwerk gesucht werden. 

\section*{Projektdokumentation}
Die Lösung des Projekts muss dokumentiert sein. 

\section*{Abgabe}
Ich benötige das Programm anfangs der Semesterwoche 15. Details zur Abgabe 
werden noch mitgeteilt. 

\section*{GUI}
Das GUI sollte folgende Funktionalitäten bieten: 
\begin{itemize}
    \item Es gibt ein Spielfeld
    \item Ich sehe, wer gerade am Zug ist. 
    \item Ich sehe, welcher Spieler mit welchen Steinen spielt
    \item Optional: Ich kann die Spielfeldgrösse ändern
\end{itemize}

\end{document}

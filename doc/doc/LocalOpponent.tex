LocalOpponent:

Diese Klasse dient zur Handhabung der künstlichen 
Intelligenz. Über die Wahl eines
Schwierigkeitsgrades lässt sich zwischen 
verschiedenen Implementationen wählen. Als Standard
wird momentan die intelligenteste Implementation 
verwendet.

Diese berechnet die Gewinnchancen für jeden Zug. 
Dazu wird auf jeder möglichen Position ein Stein
gesetzt und überprüft, ob ein Spieler gewonnen hat. 
Dies wird rekurviv für jeden Stein bis zu einer Tiefe 
von 7 Spielzügen durchgeführt. 
Ein Sieg des LocalOpponent wird dabei mit einem Wert 
von 16 bewertet. Ein Sieg des Spielers wird mit -256 
bewertet. 
Der Wert eines weiter in der Zukunft liegenden Zuges 
wird dabei durch den Faktor 2 dividiert. 
So wird ein sofortiger Sieg schwächer gewichtet als 
eine baldige Niederlage. Damit wird ein weitsichtigeres 
Spiel bezweckt. 
Von diesem Algorithmus kann auch eine Variante mit 
Threads ausgewählt werden. Dabei wird für jeden 
Stein ein eigener Thread gestartet, der dann die 
Berechung für diesen Stein durchführt. Dadurch 
können diese Berechnungen auf mehrere Prozessoren 
aufgeteilt werden. 

Ein anderer implementierter Algorithmus beachtet auch 
andere Anordnungen von Steinen, die zu einem Sieg 
führen können. Zum Beispiel werden drei Steine in 
einer Reihe mit einer Lücke mitbeachtet. Dieser wird 
jedoch nicht aktiv weiter entwickelt. 

Zudem existiert ein Algorithmus, welcher die Steine 
zufällig setzt. 

Wird kein zulässiger Algorithmus ausgewählt, werden 
die Spalten von links her aufgefüllt. 
